\documentclass{article}
\usepackage[utf8]{inputenc}
\usepackage{textcomp}

\title{Poker network protocol}
\begin{document}
\maketitle

A protocol for playing poker remotely via a text interface, usually over a network. The protocol is based on passing text commands between a server and a client. Commands should be utf-8 encoded, and each command is terminated by a newline. If the client/server receives an unrecognized command, the command should be ignored, and communication should continue as normal.

\section{Decision and card format}
\subsection{Cards}
Each card is stored as a single text token, with the suit followed by the rank. The ranks are given by the integers 2-14, the suits are given by their names (diamonds, spades, hearts, clubs). Examples: diamonds14, spades7, clubs10.

\subsection{Decisions}
A decision is stored as a single token, consisting of the type of decision and optionally the ``size'' of the decision in case of raises or bets. Possible decisions are:

\begin{itemize}
\item fold
\item check
\item call
\item smallBlind
\item bigBlindbet
\item bet\textlangle{}amount\textrangle{} 
\item raise\textlangle{}amount\textrangle{} Raises \textit{by} the amount given, not \textit{to} the amount.
\item allIn
\end{itemize}

\section{Server to client}

\subsection{upiok \textlangle{}version\textrangle{}}
Sent after it receives the initial upi command from client, to complete the init handshake.

\subsection{getName}
Requests the client name, to which it expects a name command in return.

\subsection{newgame}
Tells the client that a whole new game is being played, with new players and new stack sizes. This is not necessarily the start of the game, if the client joins the table mid-way through a game.

\subsection{amountOfPlayers \textlangle{}n\textrangle{}}
Sends the number of players on the table, including the client itself. This should be sent immediately after a newgame command, and before other information (stack sizes, blinds, names, etc) are sent.

\subsection{clientId \textlangle{}id\textrangle{}}
Sends the client's ID in the game. Every player on the table has a unique non-negative ID, which is used when sending names, positions, etc.

\subsection{playerNames \textlangle{}id1 name1\textrangle{} \textlangle{}id2 name2\textrangle{} ... }
Sends the names of all the players. This should be sent These are not used by the protocol, but are for the client to display.

\subsection{playerPositions \textlangle{}id1 position1\textrangle{} \textlangle{}id2 position2\textrangle{} ... }
Sends the position for all players' IDs. The position is sent as a non-negative integer, where position 0 is small blind, position 1 is big blind, position 2 is UTG, etc.

\subsection{stackSizes \textlangle{}id1 stackSize1\textrangle{} \textlangle{}id2 stackSize2\textrangle{} ... }
Sends the stackSizes for all players' IDs.

\subsection{getDecision \textlangle{}timeToThink\textrangle{}}
It's the client's turn, and the server is asking for a decision. The server expect a decision command in return. The timeToThink field is the time the client has to think, in milliseconds. If the client does not return a decision in time, the server may give the client a ``default'' decision, or otherwise override the client's wishes.

\section{Client to server}

\subsection{upi \textlangle{}version\textrangle{}}
Used to initialize communication over the protocol, and tell the client server that it is using unviersal poker interface. The version parameter is a sequence-based identifier. (like 1.0.0)

\subsection{name \textlangle{}name\textrangle{}}
Sends the client's name to the server. Should only be sent after the server requests it with a getName command. The name field may be empty, in which case the server may assign a name to the client. The name cannot contain whitespace characters.

\subsection{decision \textlangle{}decision\textrangle{}}
Sends the client's decision to the server. See section ``decision are card formats'' for the format for decisions.

\end{document}
 
