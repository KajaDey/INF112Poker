\documentclass{article}
\usepackage[utf8]{inputenc}
\usepackage[T1]{fontenc}
\usepackage{textcomp}

\title{Poker network protocol}
\begin{document}
\maketitle

\section{Introduction}
A protocol for playing poker remotely via a text interface, usually (but not necessarily) over a network. The protocol is based on passing text commands between the server and a client, containing all the necessary information to play the game. The server will send the settings, information about the other players, and their decisions as they are made. The client will wait for their turn, and send ack decisions after the server requests it.

Each command is terminated by a newline, and commands are separated into tokens as specified in the ``tokens'' section. The first token is always the command name. If the client or server receives a command with an unknown command name, the whole command should be ignored, and communication should continue as normal afterwards. If a known command is illegally formatted, however, the client should show an error.

\section{Tokens}
Each command consists of one or more tokens. Normally, each token is a single ``word'' (sequence of characters delimited by one or more whitespaces), but a token can also be several words inside ascii quotation marks. For example:
\begin{itemize}
\item \textbf{stackSizes 0 1000 1 1000} tokenizes to [``stacksizes'', ``0'', ``1000'', ``1'', ``1000'']
\item \textbf{name ``Svein Jonny''} tokenizes to [``name'', ``Svein Jonny'']
\end{itemize}

\section{Decision and card format}
\subsection*{Cards}
Each card is stored as a single text token (no whitespace characters), represented as the suit followed by the rank. The ranks are given by the integers 2-14, the suits are given by their names (diamonds, spades, hearts, clubs). Examples: diamonds14, spades7, clubs10.

\subsection*{Decisions}
A decision is stored as a single token (no whitespace characters), consisting of the type of decision and optionally the ``size'' of the decision in case of raises or bets. Possible decisions are:

\begin{itemize}
\item fold
\item check
\item call
\item smallBlind
\item bigBlind
\item bet\textlangle{}amount\textrangle{} 
\item raise\textlangle{}amount\textrangle{} (Raises \textit{by} the amount given, not \textit{to} the amount)
\item allIn
\end{itemize}

\section{Server to client}

\subsection*{upiok \textlangle{}version\textrangle{}}
Sent after it receives the initial upi command from client, to complete the initialization handshake. After receiving this, the client should wait for further input.

\subsection*{getName}
Requests the client's name, to which it expects a name command in return.

\subsection*{newGame}
Tells the client that a whole new game is being played, with new players and new stack sizes. 

\subsection*{newHand}
Tells the client that a new hand is starting.

\subsection*{clientId \textlangle{}ID\textrangle{}}
Sends the client's ID in the game. Each player on the table has a unique non-negative ID, which is used when sending names, positions, etc. This should be sent immediately after a newgame command, before amountOfPlayers and other information about the players.

\subsection*{amountOfPlayers \textlangle{}n\textrangle{}}
Sends the number of players on the table, including the client itself. This should be sent after clientId, but before any information about the other players (stack sizes, blinds, names, etc) is sent.

\subsection*{playerNames \textlangle{}id1 name1\textrangle{} \textlangle{}id2 name2\textrangle{} ... }
Sends the names of all the players. This should be sent at the start of each game. These are not used by the protocol, but are for the client to display.

\subsection*{playerPositions \textlangle{}id1 position1\textrangle{} \textlangle{}id2 position2\textrangle{} ... }
Sends the position for all players' IDs. This is sent at the start of each hand, i.e. when the positions change. The position is sent as a non-negative integer, where position 0 is small blind, position 1 is big blind, position 2 is UTG, etc.

\subsection*{stackSizes \textlangle{}id1 stackSize1\textrangle{} \textlangle{}id2 stackSize2\textrangle{} ... }
Sends the stackSizes for all players' IDs. This is sent at the start of each hand, and also at the end of the final hand.

\subsection*{smallBlind \textlangle{}n\textrangle{}}
Sends the small blind.

\subsection*{bigBlind \textlangle{}n\textrangle{}}
Sends the big blind.

\subsection*{levelDuration \textlangle{}n\textrangle{}}
Sends the duration of each blind level, in minutes.

\subsection*{setHand \textlangle{}ID\textrangle{} \textlangle{}card1\textrangle{} \textlangle{}card2\textrangle{}}
Sends the hole cards of the player's ID. The client's own hole cards are sent at the start of each hand; other players' hole cards are sent during showdown, or when reviewing games.

\subsection*{setFlop \textlangle{}card1\textrangle{} \textlangle{}card2\textrangle{}  \textlangle{}card3\textrangle{}}
Sends the flop cards to the client. See section ``decision and card formats'' for the format for cards.

\subsection*{setTurn \textlangle{}card\textrangle{}}
Sends the turn card to the client. See section ``decision and card formats'' for the format for cards.

\subsection*{setRiver \textlangle{}card\textrangle{}}
Sends the river card to the client. See section ``decision and card formats'' for the format for cards.

\subsection*{playerBust <ID> <rank>}
Sent when a player is busted (is out of the game). Also sends the position the player lost in.

\subsection*{showdown <line1> <line2> ... }
Sent after a showdown, sending n lines of text describing who won the pot and any side pots. These are meant to displayed in the middle of the GUI.

\subsection*{getDecision \textlangle{}timeToThink\textrangle{}}
It's the client's turn, and the server is asking for a decision. The server expect a decision command in return. The timeToThink field is the time the client has to think, in milliseconds. If the client does not return a decision in time, the server may give the client a ``default'' decision, or otherwise override the client's wishes.

\subsection*{playerMadeDecision \textlangle{}ID\textrangle{} \textlangle{}decision\textrangle{}}
Sent to every client whenever any player makes a decision in the game. Clients are also sent their own decisions this way. See section ``decision and card formats'' for the format for decisions.

\section{Client to server}

\subsection*{upi \textlangle{}version\textrangle{}}
Used to initialize communication over the protocol, and tell the client server that it is using unviersal poker interface. The version parameter is a sequence-based identifier (like 1.0.0). The client should wait for a ``upiok'' command to complete the handshake.

\subsection*{chat textToken}
Used to send chat messages to other players. The chat message is sent as a single token, so it usually needs to wrapped in quotation marks to be received properly.

\subsection*{name \textlangle{}name\textrangle{}}
Sends the client's name to the server. Should only be sent after the server requests it with a getName command. The name field may be empty, in which case the server may assign a name to the client. The name cannot contain whitespace characters.

\subsection*{decision \textlangle{}decision\textrangle{}}
Sends the client's decision to the server. See section ``decision and card formats'' for the format for decisions.

\section{Example playthrough}
\texttt{
\begin{description}
\item [Client] upi
\item [Server] upiok
\item [Server] getName
\item [Client] name Doyle
\item [Server] newGame
\item [Server] setId 2
\item [Server] amountOfPlayers 3
\item [Server] playerNames 0 Morten 1 Kristian 2 Doyle
\item [Server] playerPosition 0 0 1 1 2 2
\item [Server] stackSizes 0 5000 1 5000 2 5000
\item [Server] setHand 2 diamonds14 ace14
\item [Server] playerMadeDecision 0 smallBlind
\item [Server] playerMadeDecision 1 bigBlind
\item [Server] getDecision 5000
\item [Client] decision call
\item [Client] playerMadeDecision 2 call
\item [Server] playerMadeDecision 0 fold
\item [Server] playerMadeDecision 1 check
\item [Server] set
\end{description}
}

\end{document}
 
